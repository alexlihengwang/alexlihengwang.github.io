\documentclass{article}
\usepackage[utf8]{inputenc}

\usepackage{amssymb,amsmath,amsthm}
\usepackage{mathtools}

% Links, references
\usepackage{url}            % simple URL typesetting
\usepackage[colorlinks=true]{hyperref}

\newtheorem{theorem}{Theorem}[section]
\newtheorem{corollary}{Corollary}[section]
\newtheorem{lemma}{Lemma}[section]
\newtheorem{proposition}{Proposition}[section]

\theoremstyle{definition}
\newtheorem{definition}{Definition}[section]
\newtheorem{remark}{Remark}[section]
\newtheorem{example}{Example}[section]
\newtheorem{question}{Question}[section]
%Elementary functions
\newcommand{\floor}[1]{\left\lfloor #1 \right\rfloor}
\newcommand{\ceil}[1]{\left\lceil #1 \right\rceil}
\newcommand{\abs}[1]{\left\lvert #1 \right\rvert}
\newcommand{\by}{\times}
 
%Vector/matrix calculus
\newcommand{\norm}[1]{\left\lVert #1 \right\rVert}
\newcommand{\ip}[1]{\left\langle #1 \right\rangle}
\newcommand{\grad}{\nabla}

%Sets
\let\oldemptyset\emptyset
\let\emptyset\varnothing
\newcommand{\set}[1]{\left\{#1\right\}}

%Letters in different typefaces
%\newcommand{\A}{\mathbb{A}}
\def\A{{\mathbb{A}}}
\def\B{{\mathbb{B}}}
\def\C{{\mathbb{C}}}
\def\D{{\mathbb{D}}}
% \def\E{{\mathbb{E}}}
\def\F{{\mathbb{F}}}
\def\G{{\mathbb{G}}}
\def\H{{\mathbb{H}}}
\def\I{{\mathbb{I}}}
\def\J{{\mathbb{J}}}
\def\K{{\mathbb{K}}}
\def\L{{\mathbb{L}}}
\def\M{{\mathbb{M}}}
\def\N{{\mathbb{N}}}
\def\O{{\mathbb{O}}}
\def\P{{\mathbb{P}}}
\def\Q{{\mathbb{Q}}}
\def\R{{\mathbb{R}}}
\def\S{{\mathbb{S}}}
\def\T{{\mathbb{T}}}
\def\U{{\mathbb{U}}}
\def\V{{\mathbb{V}}}
\def\W{{\mathbb{W}}}
\def\X{{\mathbb{X}}}
\def\Y{{\mathbb{Y}}}
\def\Z{{\mathbb{Z}}}

\def\bA{{\mathbf{A}}}
\def\bB{{\mathbf{B}}}
\def\bC{{\mathbf{C}}}
\def\bD{{\mathbf{D}}}
\def\bE{{\mathbf{E}}}
\def\bF{{\mathbf{F}}}
\def\bG{{\mathbf{G}}}
\def\bH{{\mathbf{H}}}
\def\bI{{\mathbf{I}}}
\def\bJ{{\mathbf{J}}}
\def\bK{{\mathbf{K}}}
\def\bL{{\mathbf{L}}}
\def\bM{{\mathbf{M}}}
\def\bN{{\mathbf{N}}}
\def\bO{{\mathbf{O}}}
\def\bP{{\mathbf{P}}}
\def\bQ{{\mathbf{Q}}}
\def\bR{{\mathbf{R}}}
\def\bS{{\mathbf{S}}}
\def\bT{{\mathbf{T}}}
\def\bU{{\mathbf{U}}}
\def\bV{{\mathbf{V}}}
\def\bW{{\mathbf{W}}}
\def\bX{{\mathbf{X}}}
\def\bY{{\mathbf{Y}}}
\def\bZ{{\mathbf{Z}}}

\def\cA{{\mathcal A}}
\def\cB{{\mathcal B}}
\def\cC{{\mathcal C}}
\def\cD{{\mathcal D}}
\def\cE{{\mathcal E}}
\def\cF{{\mathcal F}}
\def\cG{{\mathcal G}}
\def\cH{{\mathcal H}}
\def\cI{{\mathcal I}}
\def\cJ{{\mathcal J}}
\def\cK{{\mathcal K}}
\def\cL{{\mathcal L}}
\def\cM{{\mathcal M}}
\def\cN{{\mathcal N}}
\def\cO{{\mathcal O}}
\def\cP{{\mathcal P}}
\def\cQ{{\mathcal Q}}
\def\cR{{\mathcal R}}
\def\cS{{\mathcal S}}
\def\cT{{\mathcal T}}
\def\cU{{\mathcal U}}
\def\cV{{\mathcal V}}
\def\cW{{\mathcal W}}
\def\cX{{\mathcal X}}
\def\cY{{\mathcal Y}}
\def\cZ{{\mathcal Z}}

%Optimization basics
\newcommand{\Opt}{\operatorname{Opt}}
\newcommand{\argmin}{\operatorname*{arg\,min}}
\newcommand{\argmax}{\operatorname*{arg\,max}}

%Matrix related
\newcommand{\rank}{\operatorname{rank}}
\newcommand{\Diag}{\operatorname{Diag}}
\newcommand{\diag}{\operatorname{diag}}
\newcommand{\tr}{\operatorname{tr}}
\newcommand{\range}{\operatorname{range}}
\newcommand{\codim}{\operatorname{codim}}
\newcommand{\aff}{\operatorname{aff}}
\newcommand{\spann}{\operatorname{span}}
\newcommand{\inter}{\operatorname{int}}
\newcommand{\rint}{\operatorname{rint}}
\newcommand{\cl}{\operatorname{cl}}
\newcommand{\bd}{\operatorname{bd}}
\newcommand{\rbd}{\operatorname{rbd}}

\newenvironment{smallpmatrix}
    {\left(
    \begin{smallmatrix}} 
    {\end{smallmatrix}
    \right)
    }

%Probability and distributions
\newcommand{\E}{\operatorname*{\mathbb{E}}}
\newcommand{\Var}{\operatorname{var}}

%Function related
\newcommand{\dom}{\operatorname{dom}}
\newcommand{\epi}{\operatorname{epi}}

%Convexity related
\newcommand{\conv}{\operatorname{conv}}
\newcommand{\cone}{\operatorname{cone}}
\newcommand{\clconv}{\overline{\conv}}
\newcommand{\clcone}{\overline{\cone}}

\newcommand{\Sym}{\operatorname{Sym}}
\usepackage{graphicx}
\graphicspath{ {../} }
\usepackage[numbers,sort&compress,square,comma]{natbib}
\title{Deriving the Sherman-Morrison formula}
\date{2021-01-06}
\begin{document}

In this post, I am going to motivate an intuitive derivation of the 
\href{https://en.wikipedia.org/wiki/Sherman%E2%80%93Morrison_formula}{Sherman-Morrison formula} using Schur complements.

\rule{0.5\linewidth}{\linethickness}

Recall the following fact (see also this \href{https://en.wikipedia.org/wiki/Schur_complement}{Wikipedia entry} on Schur complements).
\begin{lemma}
\label{lem:schur}
Let $M\in\R^{(n+m)\times (n+m)}$ be a block matrix with blocks
\begin{align*}
M = \begin{pmatrix}
	A & B\\
	C & D
\end{pmatrix}	.
\end{align*}
If $A$ is invertible, then there exist invertible matrices $P_1$, $Q_1\in\R^{n\by n}$ such that
\begin{align*}
P_1 M Q_1 = \begin{pmatrix}
	A - BD^{-1}C&\\& D
\end{pmatrix}.
\end{align*}
Similarly, if $D$ is invertible, then there exist invertible matrices $P_2$, $Q_2\in\R^{n\by n}$ such that
\begin{align*}
P_2 M Q_2 = \begin{pmatrix}
	A &\\&D - CA^{-1}B
\end{pmatrix}.
\end{align*}
\end{lemma}
For the purposes of this post, we won't need to know what $P_i$ and $Q_i$ look like. It is enough to know that these matrices \textit{and their inverses} have simple forms.

We will apply \cref{lem:schur} to the matrix
\begin{align*}
\begin{pmatrix}
	A & b\\
	c^\intercal & 1
\end{pmatrix},
\end{align*}
where $A\in\R^{n\by n}$ is invertible and $b$, $c\in\R^n$.
For notational convenience, let $P = P_1P_2^{-1}$ and $Q = Q_2^{-1}Q_1$.
We have that
\begin{align*}
\begin{pmatrix}
	A - bc^\intercal & \\
	& 1
\end{pmatrix} = P \begin{pmatrix}
	A & \\ & 1 - c^\intercal A^{-1}b
\end{pmatrix}Q.
\end{align*}
This identity tells us that $A-bc^\intercal$ is invertible if and only if $1 - c^\intercal A^{-1}b$ is invertible, i.e., nonzero.
Finally, inverting both sides of the equation, we have that
\begin{align*}
\begin{pmatrix}
	(A-bc^\intercal)^{-1} & \\
	& 1
\end{pmatrix} = 
Q^{-1}\begin{pmatrix}
	A^{-1} &\\
	& \tfrac{1}{1- c^\intercal A^{-1}b}
\end{pmatrix}P^{-1}.
\end{align*}
In particular, the top left block of the matrix on the right is the inverse of $A - bc^\intercal$.

\rule{0.5\linewidth}{\linethickness}

Looking back at the derivation, the main idea was to use a change of variables (the $P_i$, $Q_i$ matrices) to ``shift'' the inversion of $A - bc^\intercal$, an \textit{a priori} difficult task, to the inversion of the scalar $1 - c^\intercal A^{-1}b$, a much simpler task. This generalizes in a number of ways. For example, the same derivation applied to the $(n+m)\times(n+m)$ dimensional block matrix 
\begin{align*}
\begin{pmatrix}
	A & B\\
	C^\intercal & I_m
\end{pmatrix},
\end{align*}
where $A\in\R^{n\by n}$ is invertible and $B$, $C\in\R^{n\by m}$, gives a formula for the inverse of $A - BC^\intercal$ in terms of the inverses of $A$ and $I - C^\intercal A^{-1}B$. This latter formula is known as the Woodbury matrix identity.
\end{document}