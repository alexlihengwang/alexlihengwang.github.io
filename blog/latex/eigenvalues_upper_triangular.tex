\documentclass{article}
\usepackage[utf8]{inputenc}

\usepackage{amssymb,amsmath,amsthm}
\usepackage{mathtools}

% Links, references
\usepackage{url}            % simple URL typesetting
\usepackage[colorlinks=true]{hyperref}

\newtheorem{theorem}{Theorem}[section]
\newtheorem{corollary}{Corollary}[section]
\newtheorem{lemma}{Lemma}[section]
\newtheorem{proposition}{Proposition}[section]

\theoremstyle{definition}
\newtheorem{definition}{Definition}[section]
\newtheorem{remark}{Remark}[section]
\newtheorem{example}{Example}[section]
\newtheorem{question}{Question}[section]
%Elementary functions
\newcommand{\floor}[1]{\left\lfloor #1 \right\rfloor}
\newcommand{\ceil}[1]{\left\lceil #1 \right\rceil}
\newcommand{\abs}[1]{\left\lvert #1 \right\rvert}
\newcommand{\by}{\times}
 
%Vector/matrix calculus
\newcommand{\norm}[1]{\left\lVert #1 \right\rVert}
\newcommand{\ip}[1]{\left\langle #1 \right\rangle}
\newcommand{\grad}{\nabla}

%Sets
\let\oldemptyset\emptyset
\let\emptyset\varnothing
\newcommand{\set}[1]{\left\{#1\right\}}

%Letters in different typefaces
%\newcommand{\A}{\mathbb{A}}
\def\A{{\mathbb{A}}}
\def\B{{\mathbb{B}}}
\def\C{{\mathbb{C}}}
\def\D{{\mathbb{D}}}
% \def\E{{\mathbb{E}}}
\def\F{{\mathbb{F}}}
\def\G{{\mathbb{G}}}
\def\H{{\mathbb{H}}}
\def\I{{\mathbb{I}}}
\def\J{{\mathbb{J}}}
\def\K{{\mathbb{K}}}
\def\L{{\mathbb{L}}}
\def\M{{\mathbb{M}}}
\def\N{{\mathbb{N}}}
\def\O{{\mathbb{O}}}
\def\P{{\mathbb{P}}}
\def\Q{{\mathbb{Q}}}
\def\R{{\mathbb{R}}}
\def\S{{\mathbb{S}}}
\def\T{{\mathbb{T}}}
\def\U{{\mathbb{U}}}
\def\V{{\mathbb{V}}}
\def\W{{\mathbb{W}}}
\def\X{{\mathbb{X}}}
\def\Y{{\mathbb{Y}}}
\def\Z{{\mathbb{Z}}}

\def\bA{{\mathbf{A}}}
\def\bB{{\mathbf{B}}}
\def\bC{{\mathbf{C}}}
\def\bD{{\mathbf{D}}}
\def\bE{{\mathbf{E}}}
\def\bF{{\mathbf{F}}}
\def\bG{{\mathbf{G}}}
\def\bH{{\mathbf{H}}}
\def\bI{{\mathbf{I}}}
\def\bJ{{\mathbf{J}}}
\def\bK{{\mathbf{K}}}
\def\bL{{\mathbf{L}}}
\def\bM{{\mathbf{M}}}
\def\bN{{\mathbf{N}}}
\def\bO{{\mathbf{O}}}
\def\bP{{\mathbf{P}}}
\def\bQ{{\mathbf{Q}}}
\def\bR{{\mathbf{R}}}
\def\bS{{\mathbf{S}}}
\def\bT{{\mathbf{T}}}
\def\bU{{\mathbf{U}}}
\def\bV{{\mathbf{V}}}
\def\bW{{\mathbf{W}}}
\def\bX{{\mathbf{X}}}
\def\bY{{\mathbf{Y}}}
\def\bZ{{\mathbf{Z}}}

\def\cA{{\mathcal A}}
\def\cB{{\mathcal B}}
\def\cC{{\mathcal C}}
\def\cD{{\mathcal D}}
\def\cE{{\mathcal E}}
\def\cF{{\mathcal F}}
\def\cG{{\mathcal G}}
\def\cH{{\mathcal H}}
\def\cI{{\mathcal I}}
\def\cJ{{\mathcal J}}
\def\cK{{\mathcal K}}
\def\cL{{\mathcal L}}
\def\cM{{\mathcal M}}
\def\cN{{\mathcal N}}
\def\cO{{\mathcal O}}
\def\cP{{\mathcal P}}
\def\cQ{{\mathcal Q}}
\def\cR{{\mathcal R}}
\def\cS{{\mathcal S}}
\def\cT{{\mathcal T}}
\def\cU{{\mathcal U}}
\def\cV{{\mathcal V}}
\def\cW{{\mathcal W}}
\def\cX{{\mathcal X}}
\def\cY{{\mathcal Y}}
\def\cZ{{\mathcal Z}}

%Optimization basics
\newcommand{\Opt}{\operatorname{Opt}}
\newcommand{\argmin}{\operatorname*{arg\,min}}
\newcommand{\argmax}{\operatorname*{arg\,max}}

%Matrix related
\newcommand{\rank}{\operatorname{rank}}
\newcommand{\Diag}{\operatorname{Diag}}
\newcommand{\diag}{\operatorname{diag}}
\newcommand{\tr}{\operatorname{tr}}
\newcommand{\range}{\operatorname{range}}
\newcommand{\codim}{\operatorname{codim}}
\newcommand{\aff}{\operatorname{aff}}
\newcommand{\spann}{\operatorname{span}}
\newcommand{\inter}{\operatorname{int}}
\newcommand{\rint}{\operatorname{rint}}
\newcommand{\cl}{\operatorname{cl}}
\newcommand{\bd}{\operatorname{bd}}
\newcommand{\rbd}{\operatorname{rbd}}

\newenvironment{smallpmatrix}
    {\left(
    \begin{smallmatrix}} 
    {\end{smallmatrix}
    \right)
    }

%Probability and distributions
\newcommand{\E}{\operatorname*{\mathbb{E}}}
\newcommand{\Var}{\operatorname{var}}

%Function related
\newcommand{\dom}{\operatorname{dom}}
\newcommand{\epi}{\operatorname{epi}}

%Convexity related
\newcommand{\conv}{\operatorname{conv}}
\newcommand{\cone}{\operatorname{cone}}
\newcommand{\clconv}{\overline{\conv}}
\newcommand{\clcone}{\overline{\cone}}

\newcommand{\Sym}{\operatorname{Sym}}
\usepackage{graphicx}
\graphicspath{ {../} }
\usepackage[numbers,sort&compress,square,comma]{natbib}
\title{Eigenvalues of block matrices with upper triangular blocks}
\date{2021-01-31}
\begin{document}

This past week, \href{https://rjjiang.github.io/}{Rujun Jiang} and I posted a \href{https://arxiv.org/abs/2101.12141}{preprint} on \textit{almost} simultaneously diagonalizable quadratic forms (and other related ideas). At some point in the future, I will try write a short expository blog post about the main contributions of this work.
Today, I'm simply going to present a neat fact\footnote{I don't claim that this is new in any way and imagine this fact is known somewhere. On the other hand, I was personally unable to find a reference for it and would appreciate if anyone had pointers for me!} about the eigenvalues of block matrices with upper triangular blocks that I needed/learned/proved while working on this project.

\rule{0.5\linewidth}{\linethickness}

First, let's define the sets of matrices we care about.

\begin{definition}
A matrix $T\in\C^{n \by m}$ is \textit{upper triangular} if
\begin{align*}
m -j + i > \min(n,m) \implies T_{i,j} = 0.
\end{align*}
\end{definition}
Pictorially, an upper triangular matrix looks like
\begin{align*}
\begin{pmatrix}
	* & \cdots & *\\
	 & \ddots & \vdots\\
	 &  & *\\
	0 & \cdots & 0\\
	\vdots & \ddots & \vdots\\
	0 & \cdots & 0
\end{pmatrix}\quad\text{or}\quad
\begin{pmatrix}
	0 & \cdots & 0 & * & \cdots & *\\
	\vdots & \ddots & \vdots & & \ddots & \vdots\\
	0 & \cdots & 0 & &  & *
\end{pmatrix}
\end{align*}
when $n\geq m$ and $n\leq m$ respectively.

Suppose $(n_1,\dots,n_k)$ is such that $\sum_p n_p = n$. Given $T\in\C^{n\by n}$, we will let $[T]_{p,q}$ denote the $(p,q)$th \textit{block} of $T$ so that $[T]_{p,q} \in\C^{n_p\times n_q}$.

\begin{proposition}
Let $(n_1,\dots,n_k)$ such that $\sum_p n_p = n$ and $T\in\C^{n\by n}$ such that $[T]_{p,q}$ is upper triangular for all $p,q\in[k]$. Then, the determinant of $T$ depends only on the diagonal entries of blocks $[T]_{p,q}$ where $n_p = n_q$.
\end{proposition}

Pictorially (for the case $(n_1,n_2,n_3) = (2,2,3)$), this proposition says that the determinant of a matrix with support
% \begin{align*}
% \left(\begin{array}
% 	{c|c|c}
% 	\begin{matrix}@ & *\\& @\end{matrix} &
% 		\begin{matrix}@ & *\\& @\end{matrix} &
% 		\begin{matrix}\hphantom{@} & * & *\\&   & *\end{matrix}\\\hline
% 	\begin{matrix}@ & *\\& @\end{matrix} &
% 		\begin{matrix}@ & *\\& @\end{matrix} &
% 		\begin{matrix}\hphantom{@} & * & *\\&   & *\end{matrix}\\\hline
% 	\begin{matrix}* & *\\& *\\ & \end{matrix} &
% 		\begin{matrix}* & *\\& *\\ & \end{matrix} &
% 		\begin{matrix}@ & * & *\\& @ & *\\&& @\end{matrix}
% \end{array}\right)
% \end{align*}
\begin{align*}
\left(\begin{array}
	{cc|cc|ccc}
	@ & * & @ & * &  & * & *\\
	& @ & & @ & & & *\\\hline
	@ & * & @ & * &  & * & *\\
	& @ & & @ & & & *\\\hline
	* & * & * & * & @ & * & *\\
	& * & & * & & @ & *\\
	& & & & & & @
\end{array}\right)
\end{align*}

depends only on the entries in the $@$ positions.


At a high level, this proposition states that the determinant (whence also the characteristic polynomial and eigenvalues) of a block matrix with upper triangular blocks depends on only a very small number of entries of the block matrix.

\rule{0.5\linewidth}{\linethickness}

I will now present a really neat proof of this fact using a proof strategy that \href{http://www.cs.cmu.edu/~kpratt/}{Kevin Pratt} suggested.

\begin{proof}
We will show the stronger\footnote{This stronger statement in fact implies that the permanent, or in fact any \textit{generalized matrix function}, of such a matrix depends only on a small number of entries.} statement: For any permutation $\sigma \in S_n$, if $\prod_i T_{i,\sigma(i)}$ is nonzero, then for every $i$, we have that $(i,\sigma(i))$ is a diagonal entry of some block $[T]_{p,q}$ where $n_p = n_q$.

First assign a weight $w_i$ to each $i\in[n]$: Given $i\in[n]$, let $p$ denote the block containing $i$ and let $r$ denote position of $i$ within the $p$th block. Set
\begin{align*}
w_i \coloneqq r - n_p/2.
\end{align*}
Next for each $i,j\in[n]$, assign the weight $w_{i,j}\coloneqq w_j - w_i$.

Equivalently, if $i,j$ corresponds to $p,q,r,s$, then set
\begin{align*}
w_{i,j} \coloneqq (s- r) + (n_p + n_q)/2 - n_q.
\end{align*}

Next, as $T$ has upper triangular blocks, we also have that
\begin{align*}
T_{i,j}\neq 0 \implies (s-r) + \min(n_p,n_q) - n_q\geq 0.
\end{align*}

We deduce that if $T_{i,j}\neq 0$, then $w_{i,j}\geq 0$. Furthermore, if $T_{i,j}\neq 0$ and $w_{i,j} = 0$, then it must be the case that $n_p = n_q$ and $s = r$.


Finally, note that for any permutation $\sigma\in S_n$, we have
\begin{align*}
\sum_{i} w_{i,\sigma(i)} = \sum_i w_{\sigma(i)} - \sum_i w_i = 0.
\end{align*}

This concludes the proof as then if $\prod_i T_{i,\sigma(i)}$ is nonzero, it must be the case that for all $i\in[n]$, we have $T_{i,\sigma(i)}\neq 0$ and $w_{i,\sigma(i)} = 0$.
\end{proof}

\rule{0.5\linewidth}{\linethickness}

Intuitively, this proof looks at a matrix with upper triangular blocks and compares its support with some carefully constructed weighting of its entries. For $(n_1,n_2,n_3) = (2,2,3)$, this looks like
\begin{gather*}
\left(\begin{array}
	{cc|cc|ccc}
	* & * & * & * &  & * & *\\
	& * & & * & & & *\\\hline
	* & * & * & * &  & * & *\\
	& * & & * & & & *\\\hline
	* & * & * & * & * & * & *\\
	& * & & * & & * & *\\
	& & & & & & *
\end{array}\right),\\
\left(\begin{array}
	{cc|cc|ccc}
	0 & 1 & 0 & 1 & -0.5 & 0.5 & 1.5\\
	-1 & 0 & -1 & 0 & -1.5 & -0.5 & 0.5\\\hline
	0 & 1 & 0 & 1 & -0.5 & 0.5 & 1.5\\
	-1 & 0 & -1 & 0 & -1.5 & -0.5 & 0.5\\\hline
	0.5 & 1.5 & 0.5 & 1.5 & 0 & 1 & 2\\
	-0.5& 0.5 & -0.5 & 1.5 & -1 & 0 & 1\\
	-1.5 & -0.5 & -1.5 & -0.5 & -2 & -1 & 0
\end{array}\right).
\end{gather*}
The proof then observes that: 
The support of $T$ corresponds exactly to the entries with nonnegative weight. Additionally, the intersection of the support of $T$ with the entries of weight zero is exactly the set of diagonal entries of blocks $[T]_{p,q}$ where $n_p = n_q$.

\rule{0.5\linewidth}{\linethickness}



In the setting of our paper, we have that the blocks of $T$ are furthermore Toeplitz, i.e., their entries are constant along each diagonal. Then using the above proposition (along with other more elementary facts), we are then able to make crazy-looking statements such as
\begin{align*}
\begin{pmatrix}
	a & * & b & * & & * & *\\
	  & a & & b & & & *\\
	c & * & d & * & & * & *\\
	  & c & & d & & & *\\
	* & * & * & * & e & * & *\\
	 & * & & * & & e & *\\
	 &  & &  & &  & e\\
\end{pmatrix}
&\stackrel{\sigma}{=}
\begin{pmatrix}
	a &  & b &  & &  & \\
	  & a & & b & & & \\
	c &  & d &  & &  & \\
	  & c & & d & & & \\
	 &  &  &  & e &  & \\
	 &  & &  & & e & \\
	 &  & &  & &  & e\\
\end{pmatrix}\\
&\stackrel{\sigma}{=}
\begin{pmatrix}
	a & b&\\
	c & d&\\
	&&e
\end{pmatrix},
\end{align*}
where $\stackrel{\sigma}{=}$ indicates that the matrices have the same spectrum.
\end{document}